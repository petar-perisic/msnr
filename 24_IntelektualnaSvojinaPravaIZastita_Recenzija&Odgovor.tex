

 % !TEX encoding = UTF-8 Unicode

\documentclass[a4paper]{report} 
\usepackage[T2A]{fontenc} % enable Cyrillic fonts
\usepackage[utf8x,utf8]{inputenc} % make weird characters work
\usepackage[serbian]{babel}
%\usepackage[english,serbianc]{babel}
\usepackage{amssymb}

\usepackage{color}
\usepackage{url}
\usepackage[unicode]{hyperref}
\hypersetup{colorlinks,citecolor=green,filecolor=green,linkcolor=blue,urlcolor=blue}

\newcommand{\odgovor}[1]{\textcolor{blue}{#1}}

\begin{document}

\title{Na osnovu člana 20 zakona o autorskim i srodnim pravima ovaj naslov Vam nije dostupan
\\ \small{Strahinja Mitrić, Nemanja Gružanić, Petar Perišić, Nikola Mandić}}

\maketitle

\tableofcontents

\chapter{Recenzent \odgovor{--- ocena: 4} }

\section{O čemu rad govori?}
% Напишете један кратак пасус у којим ћете својим речима препричати суштину рада (и тиме показати да сте рад пажљиво прочитали и разумели). Обим од 200 до 400 карактера.

Osnovna tema rada je pojam intelektualne svojine i načini njene zaštite. U uvodnom delu dat je na primer plagijarizma i jedan primer iz knjige Majkla Kuina. Nakon kratkog istorijata, sledi opis stavki prava intelektualne svojine. Rad zatim opisuje softverske licence i uticaj zaštite intelektualne svojine na finansije jedne države. Završava se opisom mogućih kazni u Srbiji.

\section{Krupne primedbe i sugestije}
\label{sec:Krupne stvari}
% Напишете своја запажања и конструктивне идеје шта у раду недостаје и шта би требало да се промени-измени-дода-одузме да би рад био квалитетнији.

\subsection{Primedba na literaturu}
\label{primedba:Literatura}
U radu se kao literatura pod brojem četiri pominje rad 'Predrag Janičić , Flip Marić. Programiranje 2. Matematički Fakultet, Srbija, 2018',
ali kao što u samom radu možemo primetiti, na strani 4, na kojoj se nalaze i podaci o licenci vidimo da rad još uvek nije recenziran,
s toga smatram da se ne bi mogao naći kao zvanična literatura nekog naučnog rada, naravno dok ne bude i zvanično recenziran.

\odgovor{
	Primedba je uvažena. Izmenili smo reference na pomenutu literaturu. Umesto nje koristimo
	zvanični sajt GNU operativnog sistema i softvera otvorenog koda, pošto smo se u tekstu pozivali samo
	na deo knjige koji su govori o tipovima softvera.
}

\subsection{Primedba na primer plagijarizma}
\label{primedba:Primer plagijarizma}
U poslednjem pasusu odeljka 1.2 navodi se da će u poglavlju 3.2 biti pojašnjen ishod slučaja plagijarizma bas linije iz pesme 'Under Pressure' 
grupe Kvin i pevača Dejvid Bouvia, korišćene u pesmi 'Ice Ice Baby', repera Vanile Ajs. Međutim u odeljku 3.2 se objašnjava šta su autorska prava
i adekvatne načine zaštitite autorskih prava, dok krajnji ishod gore pomenutog slučaja plagijarizma nije objašnjen. 
\newpage

\odgovor{
	Primedba je uvažena. Nismo se lepo izrazili. Nije nam bio cilj da dalje objašnjavamo detalje slučaja plagijarizma dalje tekstu.
	Sam spor je naveden kao primer kako bi zainteresovao čitaoca i kako bi uveli pojam intelektualne svojine, kao i zaštite i prava na nju.\newline
	'Na osnovu navedenog, samo nam se nameće pitanje kako zaštititi intelektualnu svojinu? Odgovor na to pitanje
	 dajemo u poglavlju 3.2 gde bolje objašnjavamo zbog čega je Vanila Ajs morao da potpiše Dejvida Bouvija i grupu
	 Kvin kao koautore svog hita 'Ice Ice baby'. '
	'Dakle, u slučaju pesme 'Ice Ice Baby', Vanila Ajs je neautorizovano kopirao i adaptirao 
     bas liniju pesme 'Under Pressure'. Pošto je pesma bila pod autorskim pravima Dejvida Bouvija i
	 grupe 'Kvin', na kraju sudskog spora morao je da ih potpise kao koautore svog najveceg hita.'
}https://www.overleaf.com/project/5cc300c0979000312bf6d41a

\subsection{Primedba na uvod}
\label{primedba:Uvod}
Početak rečenice uvodnog dela koji glasi "Pored toga što se zapitate zašto bi iko pustio ovu pesmu", je subjektivan i nepotreban.

\odgovor{
	Primedba nije uvažena. Iako se slažemo da deo rečenice jeste subjektivan, sporni deo rečenice očigledno ne utiče na formiranje mišljenja niti čitaoca niti autora o temi o kojoj je reč, već je uveden kako bi rad bio 
	zanimljiviji za čitanje.
}

\clearpage

\section{Sitne primedbe}
% Напишете своја запажања на тему штампарских-стилских-језичких грешки

Preglednosti radi, sitne primedbe su date u tabeli \ref{tab:tabela1}.

\begin{table}[h!]
\begin{center}
\caption{Sitne primdbe}
\begin{tabular}{|c|c|c|c|c|} \hline
Odeljak & Pasus & Linija & Reč & Greška\\ \hline
1.1 & 1 & 1 & 10 & 'od' je višak\\ \hline
1.2 & 3 & 1 & 8 & veznik 'i' umesto ','\\ \hline
1.2 & 4 & 1 & 4 & 'dj' -> 'đ', zbog konzistentnosti rada\\ \hline
2 & 1 & 1 & 8 & nedostaje ','\\ \hline
3.1 & 1 & 4 & 1 & 'z' -> 'ž'\\ \hline
3.2 & 2 & 6 & 7 & 'od' je višak\\ \hline
3.4 & 2 & 2 & 3 & 'c' -> 'ć'\\ \hline
3.4 & 2 & 4 & 3 & nedostaje ' '\\ \hline
3.7 & 1 & 10 & 2 & nepotrebno 'u'\\ \hline
3.7 & 1 & 12 & 9 & 'od' je višak\\ \hline
3.7 & 2 & 2 & 3 & 'c' -> 'ć'\\ \hline
3.7 & 2 & 2 & 4 & 's' -> 'š'\\ \hline
4 & 1 & 1 & 8 & 'pojam' -> 'pojmom'\\ \hline
4 & 4 & 3 & 9 & nedostaje 'strane'\\ \hline
4.1 & 1 & 6 & 3 & nedostaje ','\\ \hline
4.2 & 3 & 14 & 4 & 'c' -> 'č'\\ \hline
5.1 & 1 & 2 & 9 & višak 'i' \\ \hline
5.2 & 1 & 4 & 7 & 'bruto' -> 'bruta' \\ \hline
6.1 & 5 & 17 & 7 & 'š' -> 'z'\\ \hline

\end{tabular}
\label{tab:tabela1}
\end{center}
\end{table}

\odgovor{
	Sve sugestije vezane za štamparske i gramatičke greške su uvažene.
}

\clearpage

\section{Provera sadržajnosti i forme seminarskog rada}
% Oдговорите на следећа питања --- уз сваки одговор дати и образложење

\begin{enumerate}
\item Da li rad dobro odgovara na zadatu temu?\\
    Da. U radu su objašnjeni osnovni pojmovi vezani za intelektualnu svojinu, dati primeri iz stvarnog života, objašnjene vrste softverskih licenci i opisano sprovođenje u Srbiji uz navedene moguće kazne.
    
\item Da li je nešto važno propušteno?\\
    Ne. Osim nedovršenog objašnjenja slučaja plagijarizma koji sam naveo u \ref{primedba:Primer plagijarizma}, mislim da su osnovne stvari dovoljno dobro objašnjene.
    
\item Da li ima suštinskih grešaka i propusta?\\
    Postoje nekoliko uočenih krupnih grešaka i propusta. Svi su navedeni u poglavlju \ref{sec:Krupne stvari}.
    
\item Da li je naslov rada dobro izabran?\\
    Da. Naslov rada je interesentan i odgovara suštini rada.

\item Da li sažetak sadrži prave podatke o radu?\\
    Da. Svi podaci koji se nalaze u sažetku, kasnije su u radu objašnjeni i razrađeni, bez propusta.

\item Da li je rad lak-težak za čitanje?\\
    Većina rada je laka i prijatna za čitanje. Jedini deo koji bih morao da izdvojim je poglavlje 6.1, koje je previše opširno i čitalac stiče utisak da je tekst samo prekopiran iz zakonika. Možda bi bilo bolje da su prekršaji i kazne navedeni u obliku tabele ili liste sa nabrajanjem.

\item Da li je za razumevanje teksta potrebno predznanje i u kolikoj meri?\\
    Ne. U radu su sve pojedinosti i svi primeri dobro objašnjeni.

\item Da li je u radu navedena odgovarajuća literatura?\\
    U velikoj meri. Jedina moguća greška je navedena u \ref{primedba:Literatura}.

\item Da li su u radu reference korektno navedene?\\
    Sve osim jedne. Sve reference su tačno navedene, odnosno ukazuju na željenu literaturu, sliku ili tabelu. Jedina greška je što se u radu nigde ne referiše na sliku broj 1.

\item Da li je struktura rada adekvatna?\\
    Da. Rad ima koncizan sažetak, dobar uvod, zanimljivu razradu punu primera i korektan zaključak.
 
\item Da li rad sadrži sve elemente propisane uslovom seminarskog rada (slike, tabele, broj strana...)?\\
    Da. Rad sadrži 2 slike, 1 tabelu, 11 strana, 9 različitih literatura (od toga bar jedna knjiga, jedan sajt i jedan članak), čime ispunjava sve zahteve.

\item Da li su slike i tabele funkcionalne i adekvatne?\\
    Da. Tabelom su precizno prikazane detaljne razlike softverskih licenci, dok se na slikama nalaze grafici koji se jasno mogu pročitati bez objašnjenja iz teksta.

\end{enumerate}

\section{Ocenite sebe}
% Napišite koliko ste upućeni u oblast koju recenzirate: 
% a) ekspert u datoj oblasti
% b) veoma upućeni u oblast
% c) srednje upućeni
d) malo upućen
% e) skoro neupućeni
% f) potpuno neupućeni
% Obrazložite svoju odluku

Kako se nalazim tek na početku svoje karijere, nisam imao puno prilika da se susrećem sa ovom temom. Međutim rad sam pažljivo pročitao, pročitao nekoliko članaka iz literature i trudio se da što korektnije recenziram rad.

\chapter{Recenzent \odgovor{--- ocena: 4} }


\section{O čemu rad govori?}
% Напишете један кратак пасус у којим ћете својим речима препричати суштину рада (и тиме показати да сте рад пажљиво прочитали и разумели). Обим од 200 до 400 карактера.
Ovaj rad govori o pojmu intelektualne svojine. Obrađena su njena prava, kao što su patenti i poslovne tajne, i mogući načini zaštite. Prikazano je kako se primenjuje u računarstvu, razlike između vrste softvera i koje sve licence za iste postoje. Dat je prikaz uticaja intelektualne svojine na ekonomiju i detaljno su objašnjene moguće kazne u našoj zemlji u slučaju piraterije i krivotvorenja. 

\section{Krupne primedbe i sugestije}
% Напишете своја запажања и конструктивне идеје шта у раду недостаје и шта би требало да се промени-измени-дода-одузме да би рад био квалитетнији.

\begin{itemize}
	\item Sekcija 1 - Smatram da je uvod preopširan i da ne bi trebalo da sadrži podsekcije. Uvod treba da bude nešto što će nas uvesti u temu Vašeg rada,
	 zaintrigirati nas i prikazati nam ono što ćete obraditi u ostataku. U poslednjem pasusu sekcije 1.2:
	\begin{displayquote}
	`` Na osnovu navedenog, samo nam se nameće pitanje kako zaštititi intelektualnu svojinu? Odgovor na to pitanje dajemo u poglavlju 3.2 gde bolje objašnjavamo 
	kako je Vanila Ajs morao da potpiše Dejvida Bouvija i grupu Kvin kao koautore svog hita `Ice Ice baby`. Ali pre toga osvrnimo se malo na istorijat.``,
	\end{displayquote}
	nam odmah dajete odgovor na to kako je spor između Vanile Ajsa i grupe Kvin rešen i na taj način nam otkrivate sve. Takođe, u sekciji 3.2 nije pomenut ovaj 
	spor i nije konkretno bolje objašnjeno zašto je Vanila Ajs potpisao kao koautore Dejvida Bouvija i grupu Kvin. Druga rečenica pasusa bi mogla da glasi ovako, 
	a time bi i opis spora bio pomenut u sekciji 3.2: 
	\begin{displayquote}
	``Odgovor na to pitanje i kako je spor između Vanile Ajsa, Dejvida Bouvija i grupe Kvin rešen dajemo u poglavlju 3.2, ali pre toga osvrnimo se malo na istorijat.``.
	\end{displayquote} 
	
	
		Jedan od načina da skratite uvod bi mogao biti da uvedete novu sekciju u okviru celog rada.
		Ona bi detaljnije objašnjavala šta je intelektualna svojina i obuhvatala većinu teksta i primere 
		iz sekcije 1.2. U ovom slučaju uvod bi mogao da sadrži kratku definiciju intelekutalne svojine i
		primer iz sekcije 1.1. Rečenica \emph{``Kako Kuin kaže `Ukoliko ukrademo nečija kola, on ih više ne
		može voziti. Ali ukoliko ukrademo nečiju šalu, obadvoje je sada možemo ispričati` ``}, bi mogla ostati u uvodu
		kao i izmenjen poslednji pasus iz sekcije 1.2. Takođe, sekcija 1.1 je nepotrebno podeljena na šest pasusa umesto da sadrži dva.  
		Sveukupno ovaj uvod bi trebalo skratiti i ne ulaziti toliko u samu temu rada, već istaći samo ključne koncepte koji će kasnije biti objašnjeni.
	
	\odgovor{
		Primedba nije uvažena. Glavna tema našeg rada je pravo i zaštita intelektualne svojine. Kako bi uveli u temu, u uvodu definišemo
		i opisujemo sam pojam intelektualne svojine. Ne smatramo da je uvod opširan, tj. da toliko ulazi u samu temu rada. Naprotiv, iako 
		uvod zauzima čitavu jednu stranu, smatramo da je zanimljiv i lako čitljiv.
		Što se tiče primera sa pesmom 'Ice Ice Baby' nije nam bila namera da se u daljem tekstu pozivamo na njega.
		Takođe nije nam bio cilj da dalje objašnjavamo detalje slučaja plagijarizma. Loše smo se izrazili u rečenici 
		i zamenili smo je. \newline
		 'Na osnovu navedenog, samo nam se nameće pitanje kako zaštititi intelektualnu svojinu? Odgovor na to pitanje
		 dajemo u poglavlju 3.2 gde bolje objašnjavamo zbog čega je Vanila Ajs morao da potpiše Dejvida Bouvija i grupu
		 Kvin kao koautore svog hita 'Ice Ice baby'. '
		'Dakle, u slučaju pesme 'Ice Ice Baby', Vanila Ajs je neautorizovano kopirao i adaptirao 
		 bas liniju pesme 'Under Pressure'. Pošto je pesma bila pod autorskim pravima Dejvida Bouvija i
		 grupe 'Kvin', na kraju sudskog spora morao je da ih potpise kao koautore svog najveceg hita.'
	}

	\item Sekcija 2 - U prvoj rečenici je potreban zarez nakon reči \emph{``patentnog``}, ali i nakon toga je rečenica konfuzna. Potrebno je bolje
		prerasporediti reči u ovoj rečenici ili izmeniti ih kako bi se bolje uočilo na sta se odnosi koji statut. Na primer, 
	\begin{displayquote}
	Monopolski statut (1624) smatra se početkom zaštite patenata, a Anin statut (1710) početkom zaštite autorskih prava.
	\end{displayquote}

	\odgovor{
		Primedba je uvažena i rečenica je preformulisana.
	}
	
	\item Sekcija 3 - U pasusu na početku ove sekcije su rečenice predugačke. Drugu i treću rečenicu je potrebno podeliti na minimum četiri ili barem koristiti zareze u okviru druge.
		\begin{itemize}
			\item Sekcija 3.1 - Petu rečenicu drugog pasusa je bolje započeti na sledeći način \emph{``Sud je 1985. godine odlučio..``}, a ne samom godinom 1985.
			\item Sekcija 3.2 - Prva rečenica nije jasna i potrebno ju je pročitati više puta kako bi se shvatilo šta ste hteli reći. Ne znam da li postoji neka gramatička greška ili ne, ali mislim da je potrebno izmeniti redosled reči ili čak celu rečenicu. Jedan način kako biste mogli to da promenite je:
			\begin{displayquote}
			Autorsko pravo u određenom vremenskom periodu štiti od neautorizovanog kopiranja ili adaptiranja umetničkih, pisanih, fotografskih i drugih oblika korišćenja Vaših ideja.
			\end{displayquote}
			Takođe, u čitavoj sekciji postoji oko 10 gramatičkih greški prilikom korišćenja zamenice Vi koju treba pisati velikim slovom kada se koristi u vidu persiranja. \\
			Koraci koji su navedeni po stavkama je bolje preraspodeliti. U drugoj stavci bih rečenicu podelila na dve tako da se prva završava posle reči \emph{``nezavisnog svedoka``}, a druga bi počinjala sa \emph{``Ta izjava svedoči..``}. Trećoj stavci bih dodala prvu rečenicu iz zagrade koja se nalazi u četvrtoj stavci zato što mislim da se više odnosi na samu treću stavku. To bi bilo urađeno na sledeći način:
			\begin{displayquote}
			Pošaljite kovertu preporučenom poštom samom sebi ili mestu za sigurno čuvanje i čuvajte poštansku potvrdu s jasnim datumom. Savetuje se da imate više od jedne koverte za slučaj da se Vaš zahtev za autorskim pravom proverava više nego jednom.
			\end{displayquote}
			Četvrta stavka bi se sastojala iz preostalih rečenica ali bez zagrada.
			
			\item Sekcija 3.4 - U naslovu ove sekcije stoji reč \emph{Trademark} koja se pominje samo u naslovu i nigde u daljem tekstu. Ovu reč je moguće iskoristi tako što se ne bi nalazila u naslovu,
			 već u prvoj rečenici. Na primer:
			\begin{displayquote}
			Zaštitni znak (žig, eng. \emph{trademark}) je reč, simbol, zvuk ili boja koju kompanija koristi za indentifikaciju robe.
			\end{displayquote}
			Potrebno je izbegavati ponavljanje istih reči u jednoj istoj rečenici, naravno ako je to moguće, pa bi u četvrtoj rečenici bilo bolje zameniti drugo pojavljivanje reči \emph{brendiranje} sa rečju \emph{ono}.

		\end{itemize}
	Takođe, sekcija 3 je duža u odnosu na ostale i mislim da je potrebno dodati odgovarajuće reference u okviru navedenih podsekcija.
		
	\odgovor{
		Sve primedbe vezane za preformulisanje rečenica i gramatičke greške su uvažene. Međutim, smatramo da gramatičke i stilske greške ne spadaju pod krupne primedbe.
	}

	\item Sekcija 4 - Mislim da ste sledeću rečenicu \emph{``Softver koji nije vlasnički, tj nevlasnički softver možemo podeliti na sledeći način:``} mogli da započnete samo sa \emph{``Nevlasnički softver``} 
	i da je početak ove rečenice samo višak. 
	Stavke koje su navedene kao nevlasnički softver mislim da su nepotrebno prevedene jer kada se u pretraživaču ukuca na primer \emph{friver softver}
	kao rezultat se ne dobija ono što treba. U ovom slučaju mislim da je bolje da koristite samo engleske nazive a da pored svake stavke napišete šta koji softver predstavlja.

	\odgovor {
		Sugestija je prihvaćena, rečenica je preformulisana.
		Što se tiče tipova nevlasničkog softvera, nismo navodili opise iz razloga što bi nepotrebno opisivali
		karakteristike svakog tipa, što i nema direktne veze sa našom temom.
	}
	
	\item Sekcija 5 - Pošto ova sekcija obuhvata dve podsekcije potrebno je imati uvodni pasus od nekoliko rečenica koji će opisati šta će biti obrađeno u okviru sekcije.
			Referenca na sliku 1 ne postoji nigde u okviru rada. Takođe, vizuelno je bolje odvojiti sliku 1 i sliku 2 jer zauzimaju čitavu stranu a odnose se na različite stvari.
	
	\odgovor{
		Primedba je uvažena. Dodali smo referencu na sliku. Takođe, sada se slike nalaze na zasebnim stranama.
		Par rečenica kao uvod u sekciju nismo dodali kako se ne bi kasnije ponavljali u tekstu.
	}

	\item Sekcija 6 - Prva dva pasusa u sekciji 6.1 treba spojiti u jedan jer se odnose na isto krivično delo. Mislim da je u ovom delu potrebna referenca zato što
		 se pozivate na članove zakona o krivičnim delima. 

	\odgovor{
		Primedba je uvažena i navedene su reference.
	}

\end{itemize}

\section{Sitne primedbe}
% Напишете своја запажања на тему штампарских-стилских-језичких грешки

\begin{enumerate}
\item Sekcija 1.1, prvi pasus, treći red: \emph{vam} $\rightarrow$ \emph{Vam}.
\item Sekcija 1.2, u okviru trećeg pasusa: primer Kuina bi bilo bolje istaći u tekstu.
\item Sekcija 1.2, četvrti pasus, prvi red: \emph{uvidjamo} $\rightarrow$ \emph{uviđamo}.
\item Sekcija 1.2, četvrti pasus, drugi red: \emph{različito} $\rightarrow$ \emph{različita}.
\item Sekcija 1.2: četvrti i peti pasus počinju isto $\rightarrow$ \emph{Na osnovu}.
\item Sekcija 2: postoji višak belina nakon godina 1769. i 1893., i potreban je zarez u poslednjoj rečenici prvog pasusa ispred reči \emph{dok}.
\item Sekcija 3.1, drugi pasus, prvi i peti red: nakon reči \emph{``instant``} potreban je razmak.
\item Sekcija 3.2, četvrta rečenica: nije potreban zarez nakon reči \emph{ideje}.
\item Sekcija 3.4, prvi pasus, četvrta rečenica: nije potreban zarez nakon prvog pojavljivanja reči \emph{brendiranje}.
\item Sekcija 3.4, drugi pasus, drugi red: \emph{ce} $\rightarrow$ \emph{će}.
\item Sekcija 3.4, drugi pasus, poslednji red:  potreban razmak posle reči \emph{termos}.
\item Sekcija 3.7, drugi pasus, drugi red: \emph{vec vise} $\rightarrow$ \emph{već više}.
\item Sekcija 4, počteak navođenja:  posle skraćenice \emph{tj} staviti tačku.
\item Sekcija 4, poslednji pasus, prvi red: \emph{pristupit} $\rightarrow$ \emph{pristupiti}.
\item Sekcija 4.2, četvrti pasus, treća rečenica: posle reči \emph{Takođe} staviti zarez.
\item Sekcija 5.1, drugi red:  \emph{svoijim} $\rightarrow$ \emph{svojim}.
\item Sekcija 5.1, poslednja rečenica: nakon reči \emph{aukcija} staviti zarez.
\item Sekcija 6.1, peti pasus, poslednji red: \emph{odušeće} $\rightarrow$ \emph{oduzeće}.
\item Kada prevodite neku reč sa engleskog jezika u zagradi se navodi \emph{eng.} umesto \emph{engl.}.
\item U nekim slučajevima je bolje umesto \emph{itd.} koristiti \emph{i slično}.
\end{enumerate}

\odgovor{
	Sve sugestije vezane za štamparske i gramatičke greške su uvažene. 
	Jedina primedba koja nije uvažena je u vezi navođenja reči na engleskom jeziku. Takođe je pravilno
	navoditi i sa \emph{engl.}.
}

\section{Provera sadržajnosti i forme seminarskog rada}
% Oдговорите на следећа питања --- уз сваки одговор дати и образложење

\begin{enumerate}
\item Da li rad dobro odgovara na zadatu temu?\\ Da, rad sadži sve bitne informacije vezane za ovu temu.
\item Da li je nešto važno propušteno?\\ Ne.
\item Da li ima suštinskih grešaka i propusta?\\
 Mislim da je propust uvod koji sadrži podsekcije i koji opisuje detaljno šta je intelektualna svojina i način zaštite iste umesto da se to nalazi u zasebnoj sekciji. 
\item Da li je naslov rada dobro izabran?\\ Da.
\item Da li sažetak sadrži prave podatke o radu?\\ Da.
\item Da li je rad lak-težak za čitanje?\\ Rad je lak za čitanje. 
\item Da li je za razumevanje teksta potrebno predznanje i u kolikoj meri?\\ Nije potrebno predznanje jer su svi ključni pojmovi lepo objašnjeni.
\item Da li je u radu navedena odgovarajuća literatura?\\ Da, ali bih dodala još jednu referencu koja bi se odnosila na sekciju 6. 
\item Da li su u radu reference korektno navedene?\\ Ne.
U uvodu se navodi primer Majkla J. Kuina i to je referenca 7. Ako navodite srednje ime autora u okviru
teksta onda je potrebno navesti ga i u literaturi. Takođe, u okviru literature obratite pažnju da prezimena autora trebaju biti sortirana i da se za imena autora navodi samo prvo slovo.
\item Da li je struktura rada adekvatna?\\ Uvod smatram da je preopširan i da bi to trebalo promeniti dodavanjem još jedne sekcije u rad. Često se delovi teksta dele na više pasusa a odnose se na jednu stvar. Ostatak rada je adekvatno urađen.
\item Da li rad sadrži sve elemente propisane uslovom seminarskog rada (slike, tabele, broj strana...)?\\ Postoji slika 1 ali je potrebna referenca i opis. Ostali propisani uslovi su ispunjeni.
\item Da li su slike i tabele funkcionalne i adekvatne?\\ Da.
\end{enumerate}

\section{Ocenite sebe}
% Napišite koliko ste upućeni u oblast koju recenzirate: 
% a) ekspert u datoj oblasti
% b) veoma upućeni u oblast
% c) srednje upućeni
% d) malo upućeni 
% e) skoro neupućeni
% f) potpuno neupućeni
% Obrazložite svoju odluku
U ovu oblast sam srednje upućena.



\chapter{Dodatne izmene}
%Ovde navedite ukoliko ima izmena koje ste uradili a koje vam recenzenti nisu tražili. 

\end{document}
